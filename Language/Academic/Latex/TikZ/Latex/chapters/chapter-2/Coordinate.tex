\section{Coordinate}
\subsection{概述}

坐标(coordinate)通常以点的形式出现,在 TikZ 中使用 () 表示一个坐标,一般语法为 

\begin{lstlisting}[style = latex]
    ([<options>] <coordinate specification>)
\end{lstlisting}
其中,coordinate specification 用于确定坐标的位置,可以使用平面直角坐标系或者极坐标系等方式确定。

主要有以下两种方式确定坐标系:
\begin{itemize}
    \item 显式(Explicitly) 
    
    可以在坐标前显式指出坐标系名,随后跟上关键字 cs(coordinate system),随后给出代表坐标位置的具体键值对。\\
    语法形式: (<coordinate system> cs: <list of key-value pairs specific to the coordinate system>)

    \begin{figure}[H]
        \centering
        \begin{minipage}{0.35\linewidth}
            \centering
            \begin{tikzpicture}[scale = 1]
                \draw (0,0) grid (3,2);
                \draw (canvas cs:x=0cm,y=2mm) 
                    -- (canvas polar cs:radius=2cm,angle=30);
            \end{tikzpicture}
        \end{minipage}
        \begin{minipage}{0.55\linewidth}
            \begin{lstlisting}[style = latex-side]
    \begin{tikzpicture}[scale = 1]
        \draw (0,0) grid (3,2);
        \draw (canvas cs:x=0cm,y=2mm) 
            -- (canvas polar cs:radius=2cm,angle=30);
    \end{tikzpicture}
            \end{lstlisting}
        \end{minipage}
        \caption{Coordinate:Explicitly}
    \end{figure}

    \item 隐式(Implicitly)

    精确说明坐标系的语法往往太过复杂,一般情况下我们更习惯与使用简化的语法,使用 (x,y) 表示平面直角坐标系点,使用 (radius,angle) 表示极坐标系。

    \begin{figure}[H]
        \centering
        \begin{minipage}{0.35\linewidth}
            \centering
            \begin{tikzpicture}[scale = 1]
                \draw (0,0) grid (3,2);
                \draw (0cm,2mm) -- (30:2cm);
            \end{tikzpicture}
        \end{minipage}
        \begin{minipage}{0.55\linewidth}
            \begin{lstlisting}[style = latex-side]
    \begin{tikzpicture}[scale = 1]
        \draw (0,0) grid (3,2);
        \draw (0cm,2mm) -- (30:2cm);
    \end{tikzpicture}
            \end{lstlisting}
        \end{minipage}
        \caption{Coordinate:Implicitly}
    \end{figure}
\end{itemize}

可以通过 [options] 给出修饰:

\begin{figure}[H]
    \centering
    \begin{minipage}{0.35\linewidth}
        \centering
        \begin{tikzpicture}[scale = 1]
            \draw       (0,0) grid (3,2);
            \draw       (0,0) -- (1,1);
            \draw[red]  (0,0) -- ([xshift=3pt] 1,1);
            \draw       (1,0) -- +(30:2cm);
            \draw[red]  (1,0) -- +([shift=(135:5pt)] 30:2cm);
        \end{tikzpicture}
    \end{minipage}
    \begin{minipage}{0.55\linewidth}
        \begin{lstlisting}[style = latex-side]
    \begin{tikzpicture}[scale = 1]
        \draw       (0,0) grid (3,2);
        \draw       (0,0) -- (1,1);
        \draw[red]  (0,0) -- ([xshift=3pt] 1,1);
        \draw       (1,0) -- +(30:2cm);
        \draw[red]  (1,0) -- +([shift=(135:5pt)] 30:2cm);
    \end{tikzpicture}
        \end{lstlisting}
    \end{minipage}
    \caption{Coordinate:修饰}
\end{figure}

\subsection{坐标系}
\subsubsection{Canvas,XYZ,Ploar 坐标系}

\begin{itemize}
    \item \textbf{canvas 坐标系}
    
    canvas 是最简单的坐标系,基本上与平面直角坐标系相同,其两个参数 x,y 分别表示偏移量 $d_x,d_y$,对应值均为 <dimension>。

    \begin{figure}[H]
        \centering
        \begin{minipage}{0.35\linewidth}
            \centering
            \begin{tikzpicture}
                \draw (0,0) grid (3,2);
                \fill (canvas cs:x=1cm,y=1.5cm) circle (2pt);
                \fill (canvas cs:x=2cm,y=-5mm+2pt) circle (2pt);
            \end{tikzpicture}
        \end{minipage}
        \begin{minipage}{0.55\linewidth}
            \begin{lstlisting}[style = latex-side]
    \begin{tikzpicture}
        \draw (0,0) grid (3,2);
        \fill (canvas cs:x=1cm,y=1.5cm) circle (2pt);
        \fill (canvas cs:x=2cm,y=-5mm+2pt) circle (2pt);
    \end{tikzpicture}
            \end{lstlisting}
        \end{minipage}
        \caption{Coordinate:canvas}
    \end{figure}

    上述显示给出了键值对,也可以隐式地只给出值。

    \item \textbf{xyz}

    xyz 坐标系类似于空间直角坐标系,在 canvas 坐标系的基础上增加了 z 轴。

    \begin{figure}[H]
        \centering
        \begin{minipage}{0.35\linewidth}
            \centering
            \begin{tikzpicture}[->]
                \draw (0,0) -- (xyz cs:x=1);
                \draw (0,0) -- (xyz cs:y=1);
                \draw (0,0) -- (xyz cs:z=1);
            \end{tikzpicture}
        \end{minipage}
        \begin{minipage}{0.55\linewidth}
            \begin{lstlisting}[style = latex-side]
    \begin{tikzpicture}[->]
        \draw (0,0) -- (xyz cs:x=1);
        \draw (0,0) -- (xyz cs:y=1);
        \draw (0,0) -- (xyz cs:z=1);
    \end{tikzpicture}
            \end{lstlisting}
        \end{minipage}
        \caption{Coordinate:xyz cs 显示写法}
    \end{figure}

    \begin{figure}[H]
        \centering
        \begin{minipage}{0.35\linewidth}
            \centering
            \begin{tikzpicture}[->]
                \draw (0,0) -- (1,0);
                \draw (0,0) -- (0,1,0);
                \draw (0,0) -- (0,0,1);
            \end{tikzpicture}
        \end{minipage}
        \begin{minipage}{0.55\linewidth}
            \begin{lstlisting}[style = latex-side]
    \begin{tikzpicture}[->]
        \draw (0,0) -- (1,0);
        \draw (0,0) -- (0,1,0);
        \draw (0,0) -- (0,0,1);
    \end{tikzpicture}
            \end{lstlisting}
        \end{minipage}
        \caption{Coordinate:xyz cs 隐式写法}
    \end{figure}

    值得说明的是 canvas 与 xyz 坐标系并没有非常明确的区分,有时候因为写法的不同,TikZ 常在这两种坐标系之间进行切换,这里仅提一下,没有必要深入了解,具体原理请参考官方手册。
    
    此外,如果我们使用 (1,0) 表示 x 方向偏移 1cm,但如果我们使用 (2+3cm,0) 的形式,默认单位则变成了 pt,其真实偏移量为 (2pt+3cm,0) 这适用于所有未显示指明单位的复合形式。

    \item\textbf{canvas polar}

    canvas polar 也即极坐标系,常用的参数有两个,angle = <degrees> 和 radius = <dimension>。其中 angle 值的范围为 [-360,720]。

    \begin{figure}[H]
        \centering
        \begin{minipage}{0.35\linewidth}
            \centering
            \begin{tikzpicture}[scale = 1]
                \draw (0cm,0cm) -- (30:1cm) -- (60:1cm) -- (90:1cm)
                        -- (120:1cm) -- (150:1cm) -- (180:1cm);
            \end{tikzpicture}
        \end{minipage}
        \begin{minipage}{0.55\linewidth}
            \begin{lstlisting}[style = latex-side]
    \begin{tikzpicture}[scale = 1]
        \draw (0cm,0cm) -- (30:1cm) -- (60:1cm) -- (90:1cm)
                -- (120:1cm) -- (150:1cm) -- (180:1cm);
    \end{tikzpicture}
            \end{lstlisting}
        \end{minipage}
        \caption{Coordinate:canvas polar}
    \end{figure}

    angle 除了给出具体的单位值,也可以使用方位 (如:north east) 来表示。

    canvas polar 还有两个不常用的值 x/y radius, 当我们给出 radius 时,可以理解为做圆得到距离,而 x/y radius 则表示做椭圆。

    \item \textbf{xyz polar}
    
    与一般极坐标系不同的是,xyz polar 坐标系将值最终转译到 xy 坐标系上,即向极坐标转换为直角坐标,这样做的目的好处包括可以定义不同 x,y 的单位长度,或者对坐标系进行变换。其值与 canvas polar 坐标系一致。

    \begin{figure}[H]
        \centering
        \begin{minipage}{0.35\linewidth}
            \centering
            \begin{tikzpicture}[x=1.5cm,y=1cm]
                \draw[help lines] (0cm,0cm) grid (3cm,2cm);

                \draw (0,0) -- (xyz polar cs:angle=0,radius=1);
                \draw (0,0) -- (xyz polar cs:angle=30,radius=1);
                \draw (0,0) -- (xyz polar cs:angle=60,radius=1);
                \draw (0,0) -- (xyz polar cs:angle=90,radius=1);

                \draw (xyz polar cs:angle=0,radius=2)
                    -- (xyz polar cs:angle=30,radius=2)
                    -- (xyz polar cs:angle=60,radius=2)
                    -- (xyz polar cs:angle=90,radius=2);
            \end{tikzpicture}
        \end{minipage}
        \begin{minipage}{0.55\linewidth}
            \begin{lstlisting}[style = latex-side]
    \begin{tikzpicture}[x=1.5cm,y=1cm]
        \draw[help lines] (0cm,0cm) grid (3cm,2cm);

        \draw (0,0) -- (xyz polar cs:angle=0,radius=1);
        \draw (0,0) -- (xyz polar cs:angle=30,radius=1);
        \draw (0,0) -- (xyz polar cs:angle=60,radius=1);
        \draw (0,0) -- (xyz polar cs:angle=90,radius=1);
        
        \draw (xyz polar cs:angle=0,radius=2)
            -- (xyz polar cs:angle=30,radius=2)
            -- (xyz polar cs:angle=60,radius=2)
            -- (xyz polar cs:angle=90,radius=2);
    \end{tikzpicture}
            \end{lstlisting}
        \end{minipage}
        \caption{Coordinate:xyz polar-单位长度}
    \end{figure}

    \begin{figure}[H]
        \centering
        \begin{minipage}{0.35\linewidth}
            \centering
            \begin{tikzpicture}[scale = 1]
                \tikz[x={(0cm,1cm)},y={(-1cm,0cm)}]
                \draw (0,0) -- (30:1) -- (60:1) -- (90:1)
                -- (120:1) -- (150:1) -- (180:1);
            \end{tikzpicture}
        \end{minipage}
        \begin{minipage}{0.55\linewidth}
            \begin{lstlisting}[style = latex-side]
    \begin{tikzpicture}[scale = 1]
        \tikz[x={(0cm,1cm)},y={(-1cm,0cm)}]
        \draw (0,0) -- (30:1) -- (60:1) -- (90:1)
        -- (120:1) -- (150:1) -- (180:1);
    \end{tikzpicture}
            \end{lstlisting}
        \end{minipage}
        \caption{Coordinate:xyz polar-y轴变换}
    \end{figure}

\end{itemize}

\subsubsection{重心坐标系}

重心坐标系(barycentric cs)通过向量 $v_1,v_2,\cdots,v_n$ 和数值 $\alpha_1,\alpha_2,\cdots,\alpha_n$ 来确定某个点位置,其计算函数如下:

\[ \frac{\alpha_1 v_1 + \alpha_2 v_2 + \cdots + \alpha_n v_n}{\alpha_1 + \alpha_2 + \cdots + \alpha_n} \]

重心坐标系需要先指定重心位置,再通过键值对赋权值。

\begin{figure}[H]
    \centering
    \begin{minipage}{0.8\linewidth}
        \centering
        \begin{tikzpicture}
            \coordinate (content) at (90:3cm);
            \coordinate (structure) at (210:3cm);
            \coordinate (form) at (-30:3cm);

            \node [above] at (content) {content oriented};
            \node [below left] at (structure) {structure oriented};
            \node [below right] at (form) {form oriented};

            \draw [thick,gray] (content.south) -- (structure.north east) -- (form.north west) -- cycle;

            \small
            \node at (barycentric cs:content=0.5,structure=0.1 ,form=1) {PostScript};
            \node at (barycentric cs:content=1 ,structure=0 ,form=0.4) {DVI};
            \node at (barycentric cs:content=0.5,structure=0.5 ,form=1) {PDF};
            \node at (barycentric cs:content=0 ,structure=0.25,form=1) {CSS};
            \node at (barycentric cs:content=0.5,structure=1 ,form=0) {XML};
            \node at (barycentric cs:content=0.5,structure=1 ,form=0.4) {HTML};
            \node at (barycentric cs:content=1 ,structure=0.2 ,form=0.8) {\TeX};
            \node at (barycentric cs:content=1 ,structure=0.6 ,form=0.8) {\LaTeX};
            \node at (barycentric cs:content=0.8,structure=0.8 ,form=1) {Word};
            \node at (barycentric cs:content=1 ,structure=0.05,form=0.05) {ASCII};
        \end{tikzpicture}
    \end{minipage}
    \begin{minipage}{0.8\linewidth}
        \begin{lstlisting}[style = latex-side]
    \begin{tikzpicture}
        \coordinate (content) at (90:3cm);
        \coordinate (structure) at (210:3cm);
        \coordinate (form) at (-30:3cm);

        \node [above] at (content) {content oriented};
        \node [below left] at (structure) {structure oriented};
        \node [below right] at (form) {form oriented};

        \draw [thick,gray] (content.south) -- (structure.north east) -- (form.north west) -- cycle;

        \small
        \node at (barycentric cs:content=0.5,structure=0.1 ,form=1) {PostScript};
        \node at (barycentric cs:content=1 ,structure=0 ,form=0.4) {DVI};
        \node at (barycentric cs:content=0.5,structure=0.5 ,form=1) {PDF};
        \node at (barycentric cs:content=0 ,structure=0.25,form=1) {CSS};
        \node at (barycentric cs:content=0.5,structure=1 ,form=0) {XML};
        \node at (barycentric cs:content=0.5,structure=1 ,form=0.4) {HTML};
        \node at (barycentric cs:content=1 ,structure=0.2 ,form=0.8) {\TeX};
        \node at (barycentric cs:content=1 ,structure=0.6 ,form=0.8) {\LaTeX};
        \node at (barycentric cs:content=0.8,structure=0.8 ,form=1) {Word};
        \node at (barycentric cs:content=1 ,structure=0.05,form=0.05) {ASCII};
    \end{tikzpicture}
        \end{lstlisting}
    \end{minipage}
    \caption{Coordinate:barycentric cs}
\end{figure}

\subsubsection{节点坐标系}

节点坐标系(node cs)主要用于获取不同节点名进行连接等操作。其主要修饰如下:

\begin{itemize}
    \item name = <node name>
    
    指定节点名
    \item anchor = <anchor>
    
    节点的锚点,可理解为连线的初始/结束点。可以省略不写,TikZ 将会自动寻找合适的绘制路径。

    \begin{figure}[H]
        \centering
        \begin{minipage}{0.7\linewidth}
            \centering
            \begin{tikzpicture}
                \node (shape) at (0,2) [draw] {|class Shape|};
                \node (rect) at (-2,0) [draw] {|class Rectangle|};
                \node (circle) at (2,0) [draw] {|class Circle|};
                \node (ellipse) at (6,0) [draw] {|class Ellipse|};

                \draw (node cs:name=circle,anchor=north) |- (0,1);
                \draw (node cs:name=ellipse,anchor=north) |- (0,1);
                \draw [arrows = -{Triangle[open, angle=60:3mm]}]
                        (node cs:name=rect,anchor=north)
                        |- (0,1) -| (node cs:name=shape,anchor=south);
            \end{tikzpicture}
        \end{minipage}
        \begin{minipage}{0.6\linewidth}
            \begin{lstlisting}[style = latex-side]
    \begin{tikzpicture}
        \node (shape) at (0,2) [draw] {|class Shape|};
        \node (rect) at (-2,0) [draw] {|class Rectangle|};
        \node (circle) at (2,0) [draw] {|class Circle|};
        \node (ellipse) at (6,0) [draw] {|class Ellipse|};
        
        \draw (node cs:name=circle,anchor=north) |- (0,1);
        \draw (node cs:name=ellipse,anchor=north) |- (0,1);
        \draw [arrows = -{Triangle[open, angle=60:3mm]}]
                (node cs:name=rect,anchor=north)
                |- (0,1) -| (node cs:name=shape,anchor=south);
    \end{tikzpicture}
            \end{lstlisting}
        \end{minipage}
        \caption{Coordinate:node cs - anchor}
    \end{figure}

    \item angle = <degree>
    
    除了通过锚点绘制线,也可以通过角度绘制。

    \begin{figure}[H]
        \centering
        \begin{minipage}{0.35\linewidth}
            \centering
            \begin{tikzpicture}
                \node (start) [draw,shape=ellipse] {start};
                \foreach \angle in {-90, -80, ..., 90}
                    \draw (node cs:name=start,angle=\angle)
                    .. controls +(\angle:1cm) and +(-1,0) .. (2.5,0);
            \end{tikzpicture}
        \end{minipage}
        \begin{minipage}{0.55\linewidth}
            \begin{lstlisting}[style = latex-side]
    \begin{tikzpicture}
        \node (start) [draw,shape=ellipse] {start};
        \foreach \angle in {-90, -80, ..., 90}
            \draw (node cs:name=start,angle=\angle)
            .. controls +(\angle:1cm) and +(-1,0) .. (2.5,0);
    \end{tikzpicture}
            \end{lstlisting}
        \end{minipage}
        \caption{Coordinate:node cs - angle}
    \end{figure}

在曲线(包括直线)绘制中,TikZ 使用 -- 表示直线,|- 表示竖直和平行线,... 表示曲线,绘制时需要用到 .. controls <contents> .. 语法。+(dx,dy) 表示偏移量。

\begin{figure}[H]
    \centering
    \begin{minipage}{0.35\linewidth}
        \centering
        \begin{tikzpicture}[scale = 1]
            \draw (0,0) node (x) [draw] {X}
                    (2,0) node (y) {Y}
                    (node cs:name=x) .. controls +(1,1) and +(-1,1) .. (node cs:name = y);
        \end{tikzpicture}
    \end{minipage}
    \begin{minipage}{0.55\linewidth}
        \begin{lstlisting}[style = latex-side]
    \begin{tikzpicture}[scale = 1]
        \draw (0,0) node (x) [draw] {X}
                (2,0) node (y) {Y}
                (node cs:name=x) .. controls +(1,1) and +(-1,1) .. (node cs:name = y);
    \end{tikzpicture}
        \end{lstlisting}
    \end{minipage}
    \caption{Coordinate:曲线绘制示例}
\end{figure}

节点坐标系作为一种经常被使用到的坐标定位方式,在实际使用中经常使用简写方式,可以省略 name 等键。

\begin{figure}[H]
    \centering
    \begin{minipage}{0.7\linewidth}
        \centering
        \begin{tikzpicture}[fill=blue!20]
            \draw[help lines] (-1,-2) grid (6,3);
            \path (0,0) node(a) [ellipse,rotate=10,draw,fill] {An ellipse}
                (3,-1) node(b) [circle,draw,fill] {A circle}
                (2,2) node(c) [rectangle,rotate=20,draw,fill] {A rectangle}
                (5,2) node(d) [rectangle,rotate=-30,draw,fill] {Another rectangle};
            \draw[thick] (a.south) -- (b) -- (c) -- (d);
            \draw[thick,red,->] (a) |- +(1,3) -| (c) |- (b);
            \draw[thick,blue,<->] (b) .. controls +(right:2cm) and +(down:1cm) .. (d);
        \end{tikzpicture}
    \end{minipage}
    \begin{minipage}{0.7\linewidth}
        \begin{lstlisting}[style = latex-side]
    \begin{tikzpicture}[fill=blue!20]
        \draw[help lines] (-1,-2) grid (6,3);
        \path (0,0) node(a) [ellipse,rotate=10,draw,fill] {An ellipse}
            (3,-1) node(b) [circle,draw,fill] {A circle}
            (2,2) node(c) [rectangle,rotate=20,draw,fill] {A rectangle}
            (5,2) node(d) [rectangle,rotate=-30,draw,fill] {Another rectangle};
        \draw[thick] (a.south) -- (b) -- (c) -- (d);
        \draw[thick,red,->] (a) |- +(1,3) -| (c) |- (b);
        \draw[thick,blue,<->] (b) .. controls +(right:2cm) and +(down:1cm) .. (d);
    \end{tikzpicture}
        \end{lstlisting}
    \end{minipage}
    \caption{Coordinate:node cs 隐式写法}
\end{figure}

\end{itemize}

\subsubsection{切线坐标系}

切线坐标系(tangent cs)需要加载 calc 包。显而易见,它是用来画切线的。其主要键如下:

\begin{itemize}
    \item node = <node>
    
    绘制切线的对象
    \item point = <point>
    
    发射切线的点
    \item solution = <number>
    
    如果有多种切线绘制方案,指定某一种
\end{itemize}

\begin{figure}[H]
    \centering
    \begin{minipage}{0.35\linewidth}
        \centering
        \begin{tikzpicture}
            \draw (0,0) grid (3,2);
            \coordinate (a) at (3,2);
            \node [circle,draw] (c) at (1,1) [minimum size=40pt] {$c$};
            \draw[red] (a) -- (tangent cs:node=c,point={(a)},solution=1) --
                (c.center) -- (tangent cs:node=c,point={(a)},solution=2) -- cycle;
        \end{tikzpicture}
    \end{minipage}
    \begin{minipage}{0.55\linewidth}
        \begin{lstlisting}[style = latex-side]
    \begin{tikzpicture}
        \draw (0,0) grid (3,2);
        \coordinate (a) at (3,2);
        \node [circle,draw] (c) at (1,1) [minimum size=40pt] {$c$};
        \draw[red] (a) -- (tangent cs:node=c,point={(a)},solution=1) --
            (c.center) -- (tangent cs:node=c,point={(a)},solution=2) -- cycle;
    \end{tikzpicture}
        \end{lstlisting}
    \end{minipage}
    \caption{Coordinate:tangent cs}
\end{figure}

\subsubsection{定义新的坐标系}

TikZ 支持定义新的坐标系,语法需要了解 TeX 底层代码,这里仅给出示例。定义新坐标系的语法格式如下:
\begin{lstlisting}[style = latex]
    \tikzdeclarecoordinatesystem{<name>}{<code>}
\end{lstlisting}

\begin{figure}[H]
    \centering
    \begin{minipage}{0.35\linewidth}
        \centering
        \makeatletter
        \define@key{cylindricalkeys}{angle}{\def\myangle{#1}}
        \define@key{cylindricalkeys}{radius}{\def\myradius{#1}}
        \define@key{cylindricalkeys}{z}{\def\myz{#1}}
        \tikzdeclarecoordinatesystem{cylindrical}{
            \setkeys{cylindricalkeys}{#1}
            \pgfpointadd{\pgfpointxyz{0}{0}{\myz}}{\pgfpointpolarxy{\myangle}{\myradius}}
        }
        \begin{tikzpicture}[z=0.2pt]
            \draw [->] (0,0,0) -- (0,0,350);
            \foreach \num in {0,10,...,350}
                \fill (cylindrical cs:angle=\num,radius=1,z=\num) circle (1pt);
        \end{tikzpicture}
    \end{minipage}
    \begin{minipage}{0.55\linewidth}
        \begin{lstlisting}[style = latex-side]
    \makeatletter
    \define@key{cylindricalkeys}{angle}{\def\myangle{#1}}
    \define@key{cylindricalkeys}{radius}{\def\myradius{#1}}
    \define@key{cylindricalkeys}{z}{\def\myz{#1}}
    \tikzdeclarecoordinatesystem{cylindrical}{
        \setkeys{cylindricalkeys}{#1}
        \pgfpointadd{\pgfpointxyz{0}{0}{\myz}}{\pgfpointpolarxy{\myangle}{\myradius}}
    }
    \begin{tikzpicture}[z=0.2pt]
        \draw [->] (0,0,0) -- (0,0,350);
        \foreach \num in {0,10,...,350}
            \fill (cylindrical cs:angle=\num,radius=1,z=\num) circle (1pt);
    \end{tikzpicture}
        \end{lstlisting}
    \end{minipage}
    \caption{Coordinate:子弟你故意坐标系}
\end{figure}

\subsection{交叉点坐标}
\subsubsection{垂线交点}

以下两个参数常用来绘制垂直于坐标轴的线段

\begin{itemize}
    \item horizontal line through = {(<coordinate>)}
    
    经过某点且平行于x轴的直线
    \item vertical line through = {(<coordinate>)}
    
    经过某点且平行于y轴的直线
\end{itemize}

以上两种方式也可以用 -| 或 |- 的形式隐式代替。

\begin{figure}[H]
    \centering
    \begin{minipage}{0.35\linewidth}
        \centering
        \begin{tikzpicture}
            \path (30:1cm) node(p1) {$p_1$} (75:1cm) node(p2) {$p_2$};

            \draw (-0.2,0) -- (1.2,0) node(xline)[right] {$q_1$};
            \draw (2,-0.2) -- (2,1.2) node(yline)[above] {$q_2$};

            \draw[->] (p1) -- (p1 |- xline);
            \draw[->] (p2) -- (p2 |- xline);
            \draw[->] (p1) -- (p1 -| yline);
            \draw[->] (p2) -- (p2 -| yline);
        \end{tikzpicture}
    \end{minipage}
    \begin{minipage}{0.55\linewidth}
        \begin{lstlisting}[style = latex-side]
    \begin{tikzpicture}
        \path (30:1cm) node(p1) {$p_1$} (75:1cm) node(p2) {$p_2$};

        \draw (-0.2,0) -- (1.2,0) node(xline)[right] {$q_1$};
        \draw (2,-0.2) -- (2,1.2) node(yline)[above] {$q_2$};
            
        \draw[->] (p1) -- (p1 |- xline);
        \draw[->] (p2) -- (p2 |- xline);
        \draw[->] (p1) -- (p1 -| yline);
        \draw[->] (p2) -- (p2 -| yline);
    \end{tikzpicture}
        \end{lstlisting}
    \end{minipage}
    \caption{Coordinate:垂线交点}
\end{figure}

注意上述的点并没有使用 (),如果比较复杂,则需要加上。

\begin{figure}[H]
    \centering
    \begin{minipage}{0.35\linewidth}
        \centering
        \begin{tikzpicture}
            \node (A) at (0,1) {A};
            \node (B) at (1,1.5) {B};
            \node (C) at (2,0) {C};
            \node (D) at (2.5,-2) {D};
            \draw (A) -- (B) node [midway] {x};
            \draw (C) -- (D) node [midway] {x};
            \node at ({$(A)!.5!(B)$} -| {$(C)!.5!(D)$}) {X};
        \end{tikzpicture}
    \end{minipage}
    \begin{minipage}{0.55\linewidth}
        \begin{lstlisting}[style = latex-side]
    \begin{tikzpicture}
        \node (A) at (0,1) {A};
        \node (B) at (1,1.5) {B};
        \node (C) at (2,0) {C};
        \node (D) at (2.5,-2) {D};
        \draw (A) -- (B) node [midway] {x};
        \draw (C) -- (D) node [midway] {x};
        \node at ({$(A)!.5!(B)$} -| {$(C)!.5!(D)$}) {X};
    \end{tikzpicture}
        \end{lstlisting}
    \end{minipage}
    \caption{Coordinate:垂线定位}
\end{figure}

\subsubsection{任意焦点的坐标}

在获取任意交点坐标之前,需要导入交点包: intersections。这将帮助我们得出交点,但是由于 TeX 底层的精度(相对专业软件)并不高,交点仅适用于绘图作为参考。

需要获得交点的线段必须被命名,以便于后续引用。相关的键如下:

\begin{itemize}
    \item name path = <name>
    
    路径名,在分号之前都有效。
    \item name path global = <name>
    
    全局路径名,在整个绘图环境中都有效。
    \item name intersections = {<options>}
    
    决定线段对象,只有值对应的曲线交点才会被计算。交点则会以 intersection-1 形式命名下去。

    \begin{figure}[H]
        \centering
        \begin{minipage}{0.35\linewidth}
            \centering
            \begin{tikzpicture}[every node/.style={opacity=1, black, above left}]
                \draw [help lines] grid (3,2);
                \draw [name path=ellipse] (2,0.5) ellipse (0.75cm and 1cm);
                \draw [name path=rectangle, rotate=10] (0.5,0.5) rectangle +(2,1);
                \fill [red, opacity=0.5, name intersections={of=ellipse and rectangle}]
                    (intersection-1) circle (2pt) node {1}
                    (intersection-2) circle (2pt) node {2};
            \end{tikzpicture}
        \end{minipage}
        \begin{minipage}{0.55\linewidth}
            \begin{lstlisting}[style = latex-side]
    \begin{tikzpicture}[every node/.style={opacity=1, black, above left}]
        \draw [help lines] grid (3,2);
        \draw [name path=ellipse] (2,0.5) ellipse (0.75cm and 1cm);
        \draw [name path=rectangle, rotate=10] (0.5,0.5) rectangle +(2,1);
        \fill [red, opacity=0.5, name intersections={of=ellipse and rectangle}]
            (intersection-1) circle (2pt) node {1}
            (intersection-2) circle (2pt) node {2};
    \end{tikzpicture}
            \end{lstlisting}
        \end{minipage}
        \caption{Coordinate:曲线交点}
    \end{figure}

    \item intersection/of = <name path 1> and <name path 2>
    
    用于指定计算交点的路径。
    \item intersection/name = <prefix>
    
    修改前缀 intersection 为对应的 prefix
    \item total = <macro>
    
    交点的数量将被记录在 TeX 的 <macro> 中。

    \begin{figure}[H]
        \centering
        \begin{minipage}{0.35\linewidth}
            \centering
            \begin{tikzpicture}
                \clip (-2,-2) rectangle (2,2);
                \draw [name path=curve 1] (-2,-1) .. controls (8,-1) and (-8,1) .. (2,1);
                \draw [name path=curve 2] (-1,-2) .. controls (-1,8) and (1,-8) .. (1,2);
                \fill [name intersections={of=curve 1 and curve 2, name=i, total=\t}]
                    [red, opacity=0.5, every node/.style={above left, black, opacity=1}]
                    \foreach \s in {1,...,\t}{(i-\s) circle (2pt) node {\footnotesize\s}};
            \end{tikzpicture}
        \end{minipage}
        \begin{minipage}{0.55\linewidth}
            \begin{lstlisting}[style = latex-side]
    \begin{tikzpicture}
        \clip (-2,-2) rectangle (2,2);
        \draw [name path=curve 1] (-2,-1) .. controls (8,-1) and (-8,1) .. (2,1);
        \draw [name path=curve 2] (-1,-2) .. controls (-1,8) and (1,-8) .. (1,2);
        \fill [name intersections={of=curve 1 and curve 2, name=i, total=\t}]
            [red, opacity=0.5, every node/.style={above left, black, opacity=1}]
            \foreach \s in {1,...,\t}{(i-\s) circle (2pt) node {\footnotesize\s}};
    \end{tikzpicture}
            \end{lstlisting}
        \end{minipage}
        \caption{Coordinate:曲线交点}
    \end{figure}

    \item by = <comma-separated list>
    
    这个键允许我们给一些特定的坐标命名,(这不影响 <prefix>-<number>中的名称)。
    
    \begin{figure}[H]
        \centering
        \begin{minipage}{0.35\linewidth}
            \centering
            \begin{tikzpicture}
                \clip (-2,-2) rectangle (2,2);
                \draw [name path=curve 1] (-2,-1) .. controls (8,-1) and (-8,1) .. (2,1);
                \draw [name path=curve 2] (-1,-2) .. controls (-1,8) and (1,-8) .. (1,2);
                \fill [name intersections={of=curve 1 and curve 2, by={a,b}}]
                    (a) circle (2pt)
                    (b) circle (2pt);
            \end{tikzpicture}
        \end{minipage}
        \begin{minipage}{0.55\linewidth}
            \begin{lstlisting}[style = latex-side]
    \begin{tikzpicture}
        \clip (-2,-2) rectangle (2,2);
        \draw [name path=curve 1] (-2,-1) .. controls (8,-1) and (-8,1) .. (2,1);
        \draw [name path=curve 2] (-1,-2) .. controls (-1,8) and (1,-8) .. (1,2);
        \fill [name intersections={of=curve 1 and curve 2, by={a,b}}]
            (a) circle (2pt)
            (b) circle (2pt);
    \end{tikzpicture}
            \end{lstlisting}
        \end{minipage}
        \caption{Coordinate:曲线交点命名}
    \end{figure}

    在命名过程中,可以使用类似 for each 中的省略。

    \begin{figure}[H]
        \centering
        \begin{minipage}{0.35\linewidth}
            \centering
            \begin{tikzpicture}
                \clip (-2,-2) rectangle (2,2);
                \draw [name path=curve 1] (-2,-1) .. controls (8,-1) and (-8,1) .. (2,1);
                \draw [name path=curve 2] (-1,-2) .. controls (-1,8) and (1,-8) .. (1,2);
                \fill [name intersections={
                    of=curve 1 and curve 2,
                    by={[label=center:a],[label=center:...],[label=center:i]}}];
            \end{tikzpicture}
        \end{minipage}
        \begin{minipage}{0.55\linewidth}
            \begin{lstlisting}[style = latex-side]
    \begin{tikzpicture}
        \clip (-2,-2) rectangle (2,2);
        \draw [name path=curve 1] (-2,-1) .. controls (8,-1) and (-8,1) .. (2,1);
        \draw [name path=curve 2] (-1,-2) .. controls (-1,8) and (1,-8) .. (1,2);
        \fill [name intersections={
            of=curve 1 and curve 2,
            by={[label=center:a],[label=center:...],[label=center:i]}}];
    \end{tikzpicture}
            \end{lstlisting}
        \end{minipage}
        \caption{Coordinate:曲线交点遍历命名}
    \end{figure}

    \item sort by=<path name>
    
    默认情形下,交点的顺序为算法计算的顺序,该键可以帮助我们按指定焦点名重新排序。

    \begin{figure}[H]
        \centering
        \begin{minipage}{0.25\linewidth}
            \centering
            \begin{tikzpicture}
                \clip (-0.5,-0.75) rectangle (3.25,2.25);
                \foreach \pathname/\shift in {line/0cm, curve/2cm}{
                    \tikzset{xshift=\shift}
                    \draw [->, name path=curve] (1,1.5) .. controls (-1,1) and (2,0.5) .. (0,0);
                    \draw [->, name path=line] (0,-.5) -- (1,2) ;
                    \fill [name intersections={of=line and curve,sort by=\pathname, name=i}]
                        [red, opacity=0.5, every node/.style={left=.25cm, black, opacity=1}]
                        \foreach \s in {1,2,3}{(i-\s) circle (2pt) node {\footnotesize\s}};
                    }
            \end{tikzpicture}
        \end{minipage}
        \begin{minipage}{0.7\linewidth}
            \begin{lstlisting}[style = latex-side]
    \begin{tikzpicture}
        \clip (-0.5,-0.75) rectangle (3.25,2.25);
        \foreach \pathname/\shift in {line/0cm, curve/2cm}{
            \tikzset{xshift=\shift}
            \draw [->, name path=curve] (1,1.5) .. controls (-1,1) and (2,0.5) .. (0,0);
            \draw [->, name path=line] (0,-.5) -- (1,2) ;
            \fill [name intersections={of=line and curve,sort by=\pathname, name=i}]
                [red, opacity=0.5, every node/.style={left=.25cm, black, opacity=1}]
                \foreach \s in {1,2,3}{(i-\s) circle (2pt) node {\footnotesize\s}};
            }
    \end{tikzpicture}
            \end{lstlisting}
        \end{minipage}
        \caption{Coordinate:交点顺序}
    \end{figure}

\end{itemize}

\subsection{相对坐标}
\subsubsection{指定相对坐标}

可以使用 ++ 来指定相对坐标(位移), ++(dx,dy) 代表在当前坐标的基础上往 x 和 y 正方向偏移 x,y 距离,这个偏移后的坐标将替换原来的坐标(类似于 C++ 传引用)。

\begin{figure}[H]
    \centering
    \begin{minipage}{0.35\linewidth}
        \centering
        \begin{tikzpicture}
            \draw (0,0) -- ++(1,0) -- ++(0,1) -- ++(-1,0) -- cycle;
            \draw (2,0) -- ++(1,0) -- ++(0,1) -- ++(-1,0) -- cycle;
            \draw (1.5,1.5) -- ++(1,0) -- ++(0,1) -- ++(-1,0) -- cycle;
        \end{tikzpicture}
    \end{minipage}
    \begin{minipage}{0.55\linewidth}
        \begin{lstlisting}[style = latex-side]
    \begin{tikzpicture}
        \draw (0,0) -- ++(1,0) -- ++(0,1) -- ++(-1,0) -- cycle;
        \draw (2,0) -- ++(1,0) -- ++(0,1) -- ++(-1,0) -- cycle;
        \draw (1.5,1.5) -- ++(1,0) -- ++(0,1) -- ++(-1,0) -- cycle;
    \end{tikzpicture}
        \end{lstlisting}
    \end{minipage}
    \caption{Coordinate:++}
\end{figure}

与 ++ 相似的还有 +,不过他不会替换原有的坐标(C++ 传值)。

\begin{figure}[H]
    \centering
    \begin{minipage}{0.35\linewidth}
        \centering
        \begin{tikzpicture}
            \draw (0,0) -- +(1,0) -- +(1,1) -- +(0,1) -- cycle;
            \draw (2,0) -- +(1,0) -- +(1,1) -- +(0,1) -- cycle;
            \draw (1.5,1.5) -- +(1,0) -- +(1,1) -- +(0,1) -- cycle;
        \end{tikzpicture}
    \end{minipage}
    \begin{minipage}{0.55\linewidth}
        \begin{lstlisting}[style = latex-side]
    \begin{tikzpicture}
        \draw (0,0) -- +(1,0) -- +(1,1) -- +(0,1) -- cycle;
        \draw (2,0) -- +(1,0) -- +(1,1) -- +(0,1) -- cycle;
        \draw (1.5,1.5) -- +(1,0) -- +(1,1) -- +(0,1) -- cycle;
    \end{tikzpicture}
        \end{lstlisting}
    \end{minipage}
    \caption{Coordinate:+}
\end{figure}

\subsubsection{旋转相对位移}

有时候我们需要获得曲线的切线或者前进方向,这时可以用 turn 键。

\begin{figure}[H]
    \centering
    \begin{minipage}{0.35\linewidth}
        \centering
        \begin{tikzpicture}[scale = 1]
            \draw (0,0) -- (1,1) -- ([turn]-45:1cm) -- ([turn]-30:1cm);
        \end{tikzpicture}
    \end{minipage}
    \begin{minipage}{0.55\linewidth}
        \begin{lstlisting}[style = latex-side]
            \tikz \draw (0,0) -- (1,1) -- ([turn]-45:1cm) -- ([turn]-30:1cm);
        \end{lstlisting}
    \end{minipage}
    \caption{Coordinate:turn}
\end{figure}

\subsubsection{相对坐标与范围}

当 scope 与 相对偏移 结合时,会发生什么?由于 scope 仅影响内部指令,坐标发生了偏移,但是路径没有发生变换;这可以帮助我们使用 scope 暂时改变坐标方位,但是又不影响原有的路径。(效果类似于 +)

一般情况下,TikZ 默认 scope 不会产生影响,即有没有 scope 没有区别。如果想要达到上述效果,需要使用 current point is local=<boolean> 键。

\begin{figure}[H]
    \centering
    \begin{minipage}{0.35\linewidth}
        \centering
        \begin{tikzpicture}
            \draw (0,0) -- ++(1,0) -- ++(0,1) -- ++(-1,0);
            \draw[red] (2,0) -- ++(1,0) { -- ++(0,1) } -- ++(-1,0);
        \end{tikzpicture}
    \end{minipage}
    \begin{minipage}{0.55\linewidth}
        \begin{lstlisting}[style = latex-side]
    \begin{tikzpicture}
        \draw (0,0) -- ++(1,0) -- ++(0,1) -- ++(-1,0);
        \draw[red] (2,0) -- ++(1,0) { -- ++(0,1) } -- ++(-1,0);
    \end{tikzpicture}
        \end{lstlisting}
    \end{minipage}
    \caption{Coordinate:scope 与 ++}
\end{figure}

\begin{figure}[H]
    \centering
    \begin{minipage}{0.35\linewidth}
        \centering
        \begin{tikzpicture}
            \draw       (0,0) -- ++(1,0) -- ++(0,1) -- ++(-1,0);
            \draw[red]  (2,0) -- ++(1,0)
                        { [current point is local] -- ++(0,1) } -- ++(-1,0);
        \end{tikzpicture}
    \end{minipage}
    \begin{minipage}{0.55\linewidth}
        \begin{lstlisting}[style = latex-side]
    \begin{tikzpicture}
        \draw       (0,0) -- ++(1,0) -- ++(0,1) -- ++(-1,0);
        \draw[red]  (2,0) -- ++(1,0)
                    { [current point is local] -- ++(0,1) } -- ++(-1,0);
    \end{tikzpicture}
        \end{lstlisting}
    \end{minipage}
    \caption{Coordinate:current point is local}
\end{figure}

\subsection{坐标计算}

此章需要使用到 calc 包。

calc 包允许我们对坐标进行一些操作(计算,增减,测量...)

下面这个例子计算出 A 点的位置并进行了位置运算绘制出红点。

\begin{figure}[H]
    \centering
    \begin{minipage}{0.35\linewidth}
        \centering
        \begin{tikzpicture}
            \draw [help lines] (0,0) grid (3,2);
            \node (a) at (1,1) {A};
            \fill [red] ($(a) + 1/3*(1cm,0)$) circle (2pt);
        \end{tikzpicture}
    \end{minipage}
    \begin{minipage}{0.55\linewidth}
        \begin{lstlisting}[style = latex-side]
    \begin{tikzpicture}
        \draw [help lines] (0,0) grid (3,2);
        \node (a) at (1,1) {A};
        \fill [red] ($(a) + 1/3*(1cm,0)$) circle (2pt);
    \end{tikzpicture}
        \end{lstlisting}
    \end{minipage}
    \caption{Coordinate:calc}
\end{figure}

\subsubsection{基础语法}

calc 包的通用语法格式如下:
\begin{lstlisting}[style = latex]
    ([<options>] $<coordinate computation>$)
\end{lstlisting}

上述语法借用了 TeX 的 \$\$ 表示数学计算。<coordinate computation> 需遵循如下结构:

\begin{lstlisting}[style = latex]
    <factor> * <coordinate> <modifiers>
\end{lstlisting}

\subsubsection{语法:运算}

TikZ 允许我们在 \$\$ 中嵌入数学运算公式(对应 factors),在写复杂公式时请使用 \{\} 区分长表达式。底层运算原理这里不做介绍,下面是几个示例:

\begin{figure}[H]
    \centering
    \begin{minipage}{0.35\linewidth}
        \centering
        \begin{tikzpicture}
            \draw [help lines] (0,0) grid (3,2);
            \fill [red] ($2*(1,1)$) circle (2pt);
            \fill [green] (${1+1}*(1,.5)$) circle (2pt);
            \fill [blue] ($cos(0)*sin(90)*(1,1)$) circle (2pt);
            \fill [black] (${3*(4-3)}*(1,0.5)$) circle (2pt);
        \end{tikzpicture}
    \end{minipage}
    \begin{minipage}{0.55\linewidth}
        \begin{lstlisting}[style = latex-side]
    \begin{tikzpicture}
        \draw [help lines] (0,0) grid (3,2);
        \fill [red] ($2*(1,1)$) circle (2pt);
        \fill [green] (${1+1}*(1,.5)$) circle (2pt);
        \fill [blue] ($cos(0)*sin(90)*(1,1)$) circle (2pt);
        \fill [black] (${3*(4-3)}*(1,0.5)$) circle (2pt);
    \end{tikzpicture}
        \end{lstlisting}
    \end{minipage}
    \caption{Coordinate:factors}
\end{figure}

\subsubsection{语法:路径修饰}

路径修饰\footnote{这里是我根据其作用起的名字}(partway modifier)语法格式如下:

\begin{lstlisting}[style = latex]
    <coordinate>!<number>!<angle>:<second coordinate>
\end{lstlisting}

以 (1,2)!.75!(3,4) 为例,它表达的是计算 (1,2) 到 (3,4) 间线段距离 3/4 的坐标。

\begin{figure}[H]
    \centering
    \begin{minipage}{0.35\linewidth}
        \centering
        \begin{tikzpicture}
            \draw [help lines] (0,0) grid (3,2);
            \draw (1,0) -- (3,2);
            \foreach \i in {0,0.2,0.5,0.9,1}
                \node at ($(1,0)!\i!(3,2)$) {\i};
        \end{tikzpicture}
    \end{minipage}
    \begin{minipage}{0.55\linewidth}
        \begin{lstlisting}[style = latex-side]
    \begin{tikzpicture}
        \draw [help lines] (0,0) grid (3,2);
        \draw (1,0) -- (3,2);
        \foreach \i in {0,0.2,0.5,0.9,1}
            \node at ($(1,0)!\i!(3,2)$) {\i};
    \end{tikzpicture}
        \end{lstlisting}
    \end{minipage}
    \caption{Coordinate:partway modifier-number}
\end{figure}

接下来引入角度(angle),以 (1,2)!.5!60:(2,2) 为例,它表示在找到中点坐标后,以第一个点为原点旋转 60 度。

\begin{figure}[H]
    \centering
    \begin{minipage}{0.35\linewidth}
        \centering
        \begin{tikzpicture}
            \draw [help lines] (0,0) grid (3,3);
            \coordinate (a) at (1,0);
            \coordinate (b) at (3,2);
            \draw[->] (a) -- (b);
            \coordinate (c) at ($ (a)!1! 10:(b) $);
            \draw[->,red] (a) -- (c);
            \fill ($ (a)!.5! 10:(b) $) circle (2pt);
        \end{tikzpicture}
    \end{minipage}
    \begin{minipage}{0.55\linewidth}
        \begin{lstlisting}[style = latex-side]
    \begin{tikzpicture}
        \draw [help lines] (0,0) grid (3,3);
        \coordinate (a) at (1,0);
        \coordinate (b) at (3,2);
        \draw[->] (a) -- (b);
        \coordinate (c) at ($ (a)!1! 10:(b) $);
        \draw[->,red] (a) -- (c);
        \fill ($ (a)!.5! 10:(b) $) circle (2pt);
    \end{tikzpicture}
        \end{lstlisting}
    \end{minipage}
    \caption{Coordinate:partway modifier-angle}
\end{figure}

来点怪的:

\begin{figure}[H]
    \centering
    \begin{minipage}{0.35\linewidth}
        \centering
        \begin{tikzpicture}
            \draw [help lines] (0,0) grid (4,4);
            \foreach \i in {0,0.1,...,2}
                \fill ($(2,2) !\i! \i*180:(3,2)$) circle (2pt);
        \end{tikzpicture}
    \end{minipage}
    \begin{minipage}{0.55\linewidth}
        \begin{lstlisting}[style = latex-side]
    \begin{tikzpicture}
        \draw [help lines] (0,0) grid (4,4);
        \foreach \i in {0,0.1,...,2}
            \fill ($(2,2) !\i! \i*180:(3,2)$) circle (2pt);
    \end{tikzpicture}
        \end{lstlisting}
    \end{minipage}
    \caption{Coordinate:partway modifier-angle}
\end{figure}

修饰可以嵌套,这样可以很好地辅助我们作图:

\begin{figure}[H]
    \centering
    \begin{minipage}{0.35\linewidth}
        \centering
        \begin{tikzpicture}
            \draw [help lines] (0,0) grid (3,2);
            \draw (0,0) -- (3,2);
            \draw[red] ($(0,0)!.3!(3,2)$) -- (3,0);
            \fill[red] ($(0,0)!.3!(3,2)!.7!(3,0)$) circle (2pt);
        \end{tikzpicture}
    \end{minipage}
    \begin{minipage}{0.55\linewidth}
        \begin{lstlisting}[style = latex-side]
    \begin{tikzpicture}
        \draw [help lines] (0,0) grid (3,2);
        \draw (0,0) -- (3,2);
        \draw[red] ($(0,0)!.3!(3,2)$) -- (3,0);
        \fill[red] ($(0,0)!.3!(3,2)!.7!(3,0)$) circle (2pt);
    \end{tikzpicture}
        \end{lstlisting}
    \end{minipage}
    \caption{Coordinate:partway modifier-嵌套}
\end{figure}

\subsubsection{语法:距离修饰}

距离修饰(distance modifier)语法格式如下:
\begin{lstlisting}[style = latex]
    <coordinate>!<dimension>!<angle>:<second coordinate>
\end{lstlisting}

可以看出,这与路径修饰语法格式几乎一致,只是将路径修饰的按比例(number)改成了按距离(dimension)。

\begin{figure}[H]
    \centering
    \begin{minipage}{0.35\linewidth}
        \centering
        \begin{tikzpicture}
            \draw [help lines] (0,0) grid (3,2);
            \draw (1,0) -- (3,2);
            \foreach \i in {0cm,1cm,15mm}
                \node at ($(1,0)!\i!(3,2)$) {\i};
        \end{tikzpicture}
    \end{minipage}
    \begin{minipage}{0.55\linewidth}
        \begin{lstlisting}[style = latex-side]
    \begin{tikzpicture}
        \draw [help lines] (0,0) grid (3,2);
        \draw (1,0) -- (3,2);
        \foreach \i in {0cm,1cm,15mm}
            \node at ($(1,0)!\i!(3,2)$) {\i};
    \end{tikzpicture}
        \end{lstlisting}
    \end{minipage}
    \caption{Coordinate:distance modifier}
\end{figure}

配合旋转修饰:

\begin{figure}[H]
    \centering
    \begin{minipage}{0.35\linewidth}
        \centering
        \begin{tikzpicture}
            \draw [help lines] (0,0) grid (3,2);
            \coordinate (a) at (1,0);
            \coordinate (b) at (3,1);
            \draw (a) -- (b);
            \coordinate (c) at ($ (a)!.25!(b) $);
            \coordinate (d) at ($ (c)!1cm!90:(b) $);
            \draw [<->] (c) -- (d) node [sloped,midway,above] {1cm};
        \end{tikzpicture}
    \end{minipage}
    \begin{minipage}{0.55\linewidth}
        \begin{lstlisting}[style = latex-side]
    \begin{tikzpicture}
        \draw [help lines] (0,0) grid (3,2);
        \coordinate (a) at (1,0);
        \coordinate (b) at (3,1);
        \draw (a) -- (b);
        \coordinate (c) at ($ (a)!.25!(b) $);
        \coordinate (d) at ($ (c)!1cm!90:(b) $);
        \draw [<->] (c) -- (d) node [sloped,midway,above] {1cm};
    \end{tikzpicture}
        \end{lstlisting}
    \end{minipage}
    \caption{Coordinate:distance modifier}
\end{figure}

\subsubsection{语法:投影修饰}

投影修饰(projection modifier)语法格式如下:
\begin{lstlisting}[style = latex]
    <coordinate>!<projection coordinate>!<angle>:<second coordinate>
\end{lstlisting}

顾名思义,投影修饰是用来绘制投影(垂线)的。以 (1,2)!(0,5)!(3,4) 为例,从(0,5)点出发,做(1,2)与(3,4)构成的直线的垂线。

\begin{figure}[H]
    \centering
    \begin{minipage}{0.35\linewidth}
        \centering
        \begin{tikzpicture}
            \draw [help lines] (0,0) grid (3,2);
            \coordinate (a) at (0,1);
            \coordinate (b) at (3,2);
            \coordinate (c) at (2.5,0);
            \draw (a) -- (b) -- (c) -- cycle;
            \draw[red] (a) -- ($(b)!(a)!(c)$);
            \draw[orange] (b) -- ($(a)!(b)!(c)$);
            \draw[blue] (c) -- ($(a)!(c)!(b)$);
        \end{tikzpicture}
    \end{minipage}
    \begin{minipage}{0.55\linewidth}
        \begin{lstlisting}[style = latex-side]
    \begin{tikzpicture}
        \draw [help lines] (0,0) grid (3,2);
        \coordinate (a) at (0,1);
        \coordinate (b) at (3,2);
        \coordinate (c) at (2.5,0);
        \draw (a) -- (b) -- (c) -- cycle;
        \draw[red] (a) -- ($(b)!(a)!(c)$);
        \draw[orange] (b) -- ($(a)!(b)!(c)$);
        \draw[blue] (c) -- ($(a)!(c)!(b)$);
    \end{tikzpicture}
        \end{lstlisting}
    \end{minipage}
    \caption{Coordinate:projection modifier}
\end{figure}


\newpage