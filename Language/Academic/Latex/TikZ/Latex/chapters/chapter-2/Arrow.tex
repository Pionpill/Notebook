\section{Arrow}
\subsection{概述}

TikZ 为我们提供了强大的 Arrow 处理功能,我们既可以使用预设的Arrow样式,自定义 Arrow,也可以自行修改预设的 Arrow。

TikZ 为我们预设的 Arrow 样式在 arrows.meta 包中。

绘制箭头大部分情况下需要满足两个基本条件:1.有终止点 2.不是闭合环路。除此之外遇到 clip ,cycle 等修饰也将无法绘制箭头。

\begin{itemize}
    \item arrows = <start arrow specification> - <end arrow specification>
    
    由于 arrows 修饰十分常用,在实际绘图中也可以省略该键,那么任何带 - 的修饰都会被认为是 arrows。

    \begin{figure}[H]
        \centering
        \begin{minipage}{0.35\linewidth}
            \centering
            \begin{tikzpicture}
                \draw[->] (0,0) -- (1,0);
                \draw[>-Stealth] (0,0.3) -- (1,0.3);
            \end{tikzpicture}
        \end{minipage}
        \begin{minipage}{0.55\linewidth}
            \begin{lstlisting}[style = latex-side]
    \begin{tikzpicture}
        \draw[->] (0,0) -- (1,0);
        \draw[>-Stealth] (0,0.3) -- (1,0.3);
    \end{tikzpicture}
            \end{lstlisting}
        \end{minipage}
        \caption{Arrow:arrows}
    \end{figure}

    上述例子十分简单,如果我们需要对箭头样式进行复杂的调整,可以使用 \{\} 添加多个修饰。

    \begin{figure}[H]
        \centering
        \begin{minipage}{0.35\linewidth}
            \centering
            \begin{tikzpicture}[scale = 1]
                \draw[-{Stealth[red]}] (0,0) -- (1,0);
            \end{tikzpicture}
        \end{minipage}
        \begin{minipage}{0.55\linewidth}
            \begin{lstlisting}[style = latex-side]
    \draw[-{Stealth[red]}] (0,0) -- (1,0);
            \end{lstlisting}
        \end{minipage}
        \caption{Arrow:arrows-\{\}}
    \end{figure}
\end{itemize}

\subsection{基本样式}
\subsubsection{箭头大小}

\begin{itemize}
    \item length = <dimension><line width factor><outer factor>
    
    这个参数控制着箭头沿发射方向的长度。

    \begin{figure}[H]
        \centering
        \begin{minipage}{0.35\linewidth}
            \centering
            \begin{tikzpicture}[scale = 1]
                \begin{scope}[yshift = 0cm]
                    \draw [-{Stealth[length=5mm]}] (0,0) -- (2,0);
                    \draw [|<->|] (1.5,.4) -- node[above=1mm] {5mm} (2,.4);
                \end{scope}
                \begin{scope}[yshift = -1.5cm]
                    \draw [-{Latex[length=5mm]}] (0,0) -- (2,0);
                    \draw [|<->|] (1.5,.4) -- node[above=1mm] {5mm} (2,.4);
                \end{scope}
            \end{tikzpicture}
        \end{minipage}
        \begin{minipage}{0.55\linewidth}
            \begin{lstlisting}[style = latex-side]
    \begin{tikzpicture}[scale = 1]
        \begin{scope}[yshift = 0cm]
            \draw [-{Stealth[length=5mm]}] (0,0) -- (2,0);
            \draw [|<->|] (1.5,.4) -- node[above=1mm] {5mm} (2,.4);
        \end{scope}
        \begin{scope}[yshift = -1.5cm]
            \draw [-{Latex[length=5mm]}] (0,0) -- (2,0);
            \draw [|<->|] (1.5,.4) -- node[above=1mm] {5mm} (2,.4);
        \end{scope}
    \end{tikzpicture}
            \end{lstlisting}
        \end{minipage}
        \caption{Arrow:length}
    \end{figure}

    下面来深入了解一下 length 的三个参数,以 Latex arrow 为例,它的 length 对应的值为 length = 3pt 4.5 0.8。默认线宽为0.4pt,最终计算的值为 $3\text{pt} + 4.5 \times 0.4\text{pt} = 4.8\text{pt}$。也即第二个参数 <line width factor> 若存在就与线宽进行乘法运算并累计入length的值。

    上面计算中第三个参数并没有起作用,原因是 <outer factor> 仅在 path 存在 double 修饰时对 inner line width 和 line width 起作用。










\end{itemize}
















\newpage