\chapter{百科}
\section{属性百科}

属性\footnote{这里借用了 css 的概念}在这里指 TikZ 中一些可选项(option)对应的通用值。

\subsection{通用属性}

通用属性,即常见的属性,其属性名和值是大部分同类语言都具备的。

\subsubsection{常见通用属性意义}

\begin{table}[H]
    \centering
    \caption{常见通用属性意义}
    \label{table:常见通用属性意义}
    \setlength{\tabcolsep}{5mm}
    \begin{tabular}{c|ccc}
        \toprule
        \textbf{属性} & \textbf{值} & \textbf{意义} & \textbf{举例}\\
        \midrule
        dimension & 数字 & 一般为长度,可自定义单位 & line width = <dimension>\\
        angle & 数字 & 旋转角度,单位为度 & rotate = <angle> \\
        scaling & 数字 & 缩放比例 & scale = <scaling  > \\
        \bottomrule
    \end{tabular}
\end{table}


\subsection{TikZ 特有属性}

特有属性,即属性和对应的值是 TikZ 所特定的,TikZ 专有这些值。

\subsubsection{对齐:alignment option }

\begin{table}[H]
    \centering
    \caption{属性:alignment option}
    \label{table:alignment option}
    \setlength{\tabcolsep}{8mm}
    \begin{tabular}{c|cc|c}
        \toprule
        \textbf{值} & \textbf{意义} & \textbf{值} & \textbf{意义} \\
        \midrule
        left & 左对齐 & flush left & 左对齐(禁止拆分单词) \\
        right & 右对齐 & flush right & 右对齐(禁止拆分单词) \\
        center & 居中对齐 & flush center & 居中对齐(禁止拆分单词) \\
        justify & 拉长 & none & 禁止之前的对齐属性 \\
        \bottomrule
    \end{tabular}
\end{table}