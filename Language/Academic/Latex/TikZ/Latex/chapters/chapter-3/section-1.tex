\chapter{百科}
\section{参数百科}

参数\footnote{这里借用了 css 的概念}在这里指 TikZ 中一些可选项(option)对应的通用值。

\subsection{通用参数}

通用参数,即常见的参数,其参数名和值是大部分同类语言都具备的,只需理解参数意思就可明白如果写值。

\subsubsection{常见通用参数意义}

\begin{table}[H]
    \centering
    \caption{常见通用参数意义}
    \label{table:常见通用参数意义}
    \setlength{\tabcolsep}{5mm}
    \begin{tabular}{c|ccc}
        \toprule
        \textbf{参数} & \textbf{值} & \textbf{意义} & \textbf{举例}\\
        \midrule
        dimension & 数字 & 一般为长度,可自定义单位 & line width = <dimension>\\
        angle & 数字 & 旋转角度,单位为度 & rotate = <angle> \\
        scaling & 数字 & 缩放比例 & scale = <scaling  > \\
        \bottomrule
    \end{tabular}
\end{table}


\subsection{TikZ 特有参数}

特有参数,即参数和对应的值是 TikZ 所特定的,TikZ 专有这些值,参数的值往往需要查表得知。

\subsubsection{对齐:alignment option }

\begin{table}[H]
    \centering
    \caption{参数:alignment option}
    \label{table:alignment option}
    \setlength{\tabcolsep}{8mm}
    \begin{tabular}{c|cc|c}
        \toprule
        \textbf{值} & \textbf{意义} & \textbf{值} & \textbf{意义} \\
        \midrule
        left & 左对齐 & flush left & 左对齐(禁止拆分单词) \\
        right & 右对齐 & flush right & 右对齐(禁止拆分单词) \\
        center & 居中对齐 & flush center & 居中对齐(禁止拆分单词) \\
        justify & 拉长 & none & 禁止之前的对齐参数 \\
        \bottomrule
    \end{tabular}
\end{table}

\subsubsection{锚点:anchor name}

锚点主要用在定位与文本对齐上,锚点的值需要视具体的修饰选择。

\begin{table}[H]
    \centering
    \caption{锚点:anchor name}
    \label{table:锚点:anchor name}
    \setlength{\tabcolsep}{3.8mm}
    \begin{threeparttable}
        \begin{tabular}{c|cc|cc}
            \toprule
            \textbf{类型} & \textbf{值} & \textbf{意义} & \textbf{值} & \textbf{意义}\\
            \midrule
            \multirow{2}{*}{对齐}   & center & 文字中心对齐\tnote{1} & base & 文字底部对齐\tnote{2} \\
                                    & mid & 文本中心对齐\tnote{3} &  &  \\
            \midrule
            \multirow{4}{*}{位置}   & north & 北 & south & 南 \\
                                    & east & 东 & west & 西 \\
                                    & north-east & 东北 & south-east & 东南 \\
                                    & north-west & 西北 & south-west & 西南 \\
            \midrule
            \multirow{2}{*}{组合}   & base west & 采用 base 对齐的西方 & base east & 采用 base 对齐的东方\\
                                    & mid west & 采用 mid 对齐的西方 & mid east & 采用 mid 对齐的东方\\
            \bottomrule
        \end{tabular}
        \begin{tablenotes}
            \footnotesize
            \item[1] 单个文字的中心。
            \item[2] 文字底边对齐,英文四线三格中第三条线对齐。
            \item[3] 文本整体的中心。
        \end{tablenotes}
    \end{threeparttable}
\end{table}

\subsubsection{偏移:offset}

\begin{table}[H]
    \centering
    \caption{偏移:offset}
    \label{table:偏移:offset}
    \setlength{\tabcolsep}{9mm}
    \begin{threeparttable}
        \begin{tabular}{c|cc|c}
            \toprule
            \textbf{值} & \textbf{意义} & \textbf{值} & \textbf{意义}\\
            \midrule
            above/below & 上/下 & left/right & 左/右 \\
            above left/right & 上左/右 & below left/right & 下左/右  \\
            centered & 中心 & & \\
            \bottomrule
        \end{tabular}
    \end{threeparttable}
\end{table}