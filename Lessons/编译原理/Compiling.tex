\documentclass{PionpillNote-art}

\import{sections}{style.tex}

\title{编译原理笔记}
\author{
    Pionpill \footnote{笔名:北岸,电子邮件:673486387@qq.com,Github:\url{https://github.com/Pionpill}} \\
    本文档为作者学习《编译原理》\footnote{《程序设计语言 编译原理》:陈火旺,国防工业出版社,2020年印刷}一书时的笔记,\\
}

\date{\today}

\begin{document}

\maketitle

\noindent\textbf{前言:}

本篇笔记用于应付学校期末考试,本人对这方面也没有进行深入研究。内容浅显,不适合考研等深入学习。

此外,本篇笔记是对原书的提炼总结,只能辅助原书进行学习,有大量例子等理解性文字并未进行说明,如有需要请购买原书。本笔记文案多为原书摘抄或个人总结,图片为本人使用 TikZ 绘制,若需进行引用,可前往下载 \LaTeX 源代码\footnote{\url{https://github.com/Pionpill/Notebook/tree/Pionpill/Lessons}},请遵守 GPL-v3 协议。

\date{\today}

\tableofcontents
\thispagestyle{empty}
\newpage
\setcounter{page}{1} 

\import{sections}{section-1.tex}
\import{sections}{section-2.tex}
\import{sections}{appendix.tex}




\end{document}

