\section{词法分析}
\subsection{对于词法分析器的要求}
\subsubsection{词法分析器的功能和输出形式}

词法分析器的功能是输入源程序,输出单词符号。程序语言的单词符号一般可分为下列五种。

\begin{enumerate}
    \item \textbf{关键字}:由程序语言定义的具有固定意义的标识符。有时被称为保留字或基本字。例如 begin,if,while......这些字通常不用作一般标识符。
    \item \textbf{标识符}:用来表示各种名字,如变量名,数组名,过程名等。
    \item \textbf{常数}:一般用整型,实型,布尔型等等
    \item \textbf{运算符}:如 +,-,*,/ 等
    \item \textbf{界符}:如逗号,括号,分号,注释符等
\end{enumerate}

词法分析器所输出的单词符号常常表示为以下二元式:
\begin{center}
    (单词种别,单词符号的属性值)
\end{center}

单词种别通常采用整数编码。如果一个种别只含一个单词符号,那么,对于这个单词符号,种别编码就完全代表它自身了。若一个种别含有多个单词符号,那么,对于它的每个单词符号,除了给出种别编码之外,还应给出有关的单词符号的属性信息。

单词符号的属性是指单词符号的特征或特性。属性值则是反应特性或特征的值。例如,对于某个标识符,常将存放它的有关信息的符号表项的指针作为属性值。

考虑如下 C++ 代码段:
\begin{center}
    while(i>=j) i--;
\end{center}

经词法分析器处理后,它将被转换为如下的单词符号序列:
\begin{center}
    \begin{code}
        <while,->
        <(,->
        <id,指向i的符号表项的指针>
        <>=,->
        <id,指向j的符号表项的指针>
        <),->
        <id,指向i的符号表项的指针>
        <--,->
        <;,->
    \end{code}
\end{center}

\subsubsection{词法分析器作为一个独立子程序}

把词法分析器安排为一个独立阶段的好处是:它可以使整个编译程序的结构更简洁,清晰和条理化。

我们可以把词法分析器安排成一个子程序,每当语法分析器需要一个单词符号时就调用这个子程序。每一次调用,词法分析器就从输入串中识别出一个单词符号,把它交给语法分析器。后文假定词法分析器按这种方式工作。

\subsection{词法分析器的设计}
\subsubsection{输入,预处理}

词法分析器工作的第一步是输入源程序文本。输入串一般是放在一个缓冲区中,称为输入缓冲区。词法分析器的工作(单词符号的识别)可以直接在这个缓冲区中进行。在多数情况下,会对输入串进行预处理。

对于许多程序语言来说,空白符,跳格符,回车符,换行符等编辑性字符除了在文字常数中之外,在别处的任何出现都没有意义,而注解部分几乎允许出现在程序中的任何地方。它们不是程序语言的必要组成部分,它们存在的意义仅是提高程序的易读性和易理解性。对于它们,预处理可以将其剔掉。

有些语言将空白符(一个或数个)用作单词符号之间的间隔,即界符。这种情况下,预处理时可把相继的若干个空白结合成一个。

我们可以设想构造一个预处理子程序,它能够完成上面所述的任务。每当词法分析器调用它时,它就处理出一窜确定长度的输入自读,并将其装进词法分析器所指定的缓冲区中(扫描缓冲区)。这样分析器就可以在此缓冲区中直接进行单词符号的识别,而不必照管其他繁琐事务。

分析器对扫描缓冲区进行扫描时一般用两个指示器,一个指向当前正在识别的单词的开始位置(新单词首字符),另一个用于向前搜索以寻找单词的终点。

无论扫描缓冲区设得多大都不能保证单词符号不会被它的边界所打断。因此,扫描缓冲区最好的使用一个如下所示得一分为二得区域。
\begin{figure}[H]
    \centering
    \begin{tikzpicture}[font = \small]
        \draw (0,0) rectangle (4,0.5);
        \draw (4,0) rectangle (8,0.5);
        \node (begin) at (5,-1) {起点指示器};
        \node (search) at (7,-1) {搜索指示器};
        \draw [-Stealth] (begin) -- (5,0);
        \draw [-Stealth] (search) -- (7,0);
    \end{tikzpicture}
    \caption{扫描缓冲区}
    \label{扫描缓冲区}
\end{figure}
假定每半区可容120个字符,而这两个半区又是互补使用的。如果收缩指示器从单词起点出发搜索到半区得边缘但尚未到达单词得终点,那么就应调用预处理程序,令其把后续的120个输入字符装进另半区。我们认定,在搜索指示器对另半区进行扫描得时期内,线性单词得终点必能够达到。这意味着对标识符和常数得长度实际上必须加以限制(例如不多于120字符)。

\subsubsection{单词符号得识别:超前搜索}

词法分析器得结构如图\ref{词法分析器}所示,当词法分析器调用预处理子程序处理出一串输入字符放进扫描缓冲区之后,分析器就从此缓冲器中逐一识别单词符号。当缓冲区里得字符串被处理完之后,他又调用预处理程序装入新串。

下面介绍单词符号识别的一个简单方法 —— 超前搜索。

\textbf{关键字的识别}\footnote{书P39-40有一个FORTRAN语言的例子}

在一些关键字不加以保护(用户可以将关键字用作普通标识符)的语言中,为了识别关键字,我们必须超前扫描多个字符,超前到能够确定词性的地方为止。


\begin{figure}[H]
    \centering
    \begin{tikzpicture}[font = \small,thick]
        \begin{scope}[inner xsep = 3ex,inner ysep=1ex,text width=3em]
            \node[draw] (process) at (0,0) {预处理子程序};
            \node[draw] (scanner) at (0,-2) {扫描器};
            \node[draw,text width = 5em] (scanner-buffer) at (5,-1) {扫描缓冲区};
            \node[draw,text width = 5em,inner ysep = 3ex] (input-buffer) at (5,1) {输入缓冲区};
            \node[draw,text width=2em] (list) at (9,1) {列表};
        \end{scope}
        \node[draw,ellipse] (input) at (5,3) {输入};
        \begin{scope}[-Stealth,dashed]
            \draw (input) -- (input-buffer);
            \draw (input-buffer) -- (list);
            \draw (input-buffer.west) -- ++(0,-0.6) -- ++ (-2.2,0);
            \draw (process) -| (scanner-buffer);
            \draw (scanner-buffer) |- (scanner);
        \end{scope}
        \begin{scope}[-Stealth]
            \draw (process.north) -- ++ (0,1) -- ++ (0,-1);
            \draw (process.south) -- ++ (-0.5,0) -- ++ (0,-1);
            \draw (scanner.north) -- ++ (+0.5,0) -- ++ (0,1);
            \draw (scanner.south) -- ++ (0,-1) node[left,pos=0.5] {单词符号};
        \end{scope}
    \end{tikzpicture}
    \caption{词法分析器}
    \label{词法分析器}
\end{figure}

\textbf{标识符的识别}

多数语言的标识符时字母开头的``字母/数字''串,而且在程序中标识符的出现都后跟着算符或界符,因此识别标识符大多没有困难。

\textbf{常数的识别}

多数语言算术常数的表示大体相似,对它们的识别也是很直接的。但对于某些语言的常数的识别也需要超前搜索的方法。

逻辑(或布尔)常数和用引用括起来的字符串常数都很容易识别。但 FORTRAN 有需格外处理。

\textbf{算符和界符的识别}

词法分析器应将那些由多个字符复合成的算符和界符拼合成一个单词符号。因为这些字符串是不可分的整体,在这里同样需要超前搜索。

\subsubsection{状态转换图}

状态转换图是设计词法分析程序的一种好途径。转换图是一张有限方向图。在状态转换图中,结点代表状态,用圆圈表示。状态之间用箭弧连结。箭弧上的标记(字符)代表在射出结点状态下可能出现的输入字符或字符类。

\begin{figure}[H]
    \centering
    \subfigure[]{
        \begin{minipage}{0.3\linewidth}
            \centering
            \begin{tikzpicture}[font=\small,shape=circle,-Stealth,thick]
                \node[draw] (1) at (0,0) {1};
                \node[draw] (2) at (3,0) {2};
                \node[draw] (3) at (3,-1.5) {3};
                \draw (1) -- (2) node [pos=0.5,above] {X};
                \draw (1) to[out = 270,in = 180] node [pos=0.5,above] {Y} (3);
            \end{tikzpicture}
        \end{minipage}
    }
    \subfigure[]{
        \begin{minipage}{0.3\linewidth}
            \centering
            \begin{tikzpicture}[font=\small,shape=circle,-Stealth,thick]
                \node[draw] (0) at (0,0) {0};
                \node[draw] (1) at (2,0) {1};
                \node[draw,double] (2) at (4,0) {2};
                \draw (0) -- (1) node[above=-8pt,pos=0.5] {字母};
                \draw (1) -- (2) node[above=-8pt,pos=0.5] {其它};
                \draw (1) ..controls +(0.5,1.2) and + (-0.5,1.2).. (1) node [above = -0.7cm,pos=0.5] {字母或数字};
            \end{tikzpicture}
        \end{minipage}

    }
    \subfigure[]{
        \begin{minipage}{0.3\linewidth}
            \centering
            \begin{tikzpicture}[font=\small,shape=circle,-Stealth,thick]
                \node[draw] (0) at (0,0) {0};
                \node[draw] (1) at (2,0) {1};
                \node[draw,double] (2) at (4,0) {2};
                \draw (0) -- (1) node[above=-8pt,pos=0.5] {数字};
                \draw (1) -- (2) node[above=-8pt,pos=0.5] {其它};
                \draw (1) ..controls +(0.5,1.2) and + (-0.5,1.2).. (1) node [above = -0.2cm, pos=0.5] {数字};
            \end{tikzpicture}
        \end{minipage}

    }
    \caption{状态转换图}
\end{figure}









\newpage