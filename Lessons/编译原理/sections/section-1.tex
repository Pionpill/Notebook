\newpage
\section{引论}
\subsection{什么是编译原理}

计算机上执行一个高级语言程序通常分为两步:第一步,用一个编译程序把高级语言翻译成机器语言程序;第二部,运行所得到的机器语言程序求计算结果。

通常所说的翻译程序是这样一个程序,它能够把某一种语言程序(称为源语言程序)转换成另一种语言程序(称为目标语言程序),前后者逻辑上是等价的。这样的一个翻译程序就称为编译程序。

高级语言除了先编译后执行外,有时也可``解释''执行。一个源语言的解释程序就是这样的程序,它以该语言写的源程序作为输入,但不产生目标程序,而是边解释边执行源程序本身。

\subsection{编译过程概述}

编译程序的工作过程一般可分为五个阶段:词法分析,语法分析,语义分析与中间代码产生,优化,目标代码生成。

\noindent\textbf{词法分析}

词法分析的任务是:输入源程序,对构成源程序的字符串进行扫描和分解,识别出一个个单词。这一过程类似于英文翻译中认识每一个单词的意义。

\noindent\textbf{语法分析}

语法分析的任务是:在词法分析的基础上,根据语言的语法规则,把单词符号串分解成各类语法单位,如``短语'',``子句'',``句子'',``程序段''等。通过语法分析,确定整个输入串是否构成语法上正确的``程序''。语法分析所依循的是语言的语法规则。语法规则通常用上下文无关文法描述。词法分析是一种线性分析,而语法分析是一种层次结构分析。

\noindent\textbf{语义分析和中间代码产生} (例子见书 P3)

这一阶段的任务是:对语法分析所识别出的各类语法范畴,分析其含义,并进行初步翻译(产生中间代码)。这一阶段通常包括两个方面的工作。首先,对每种语法范畴进行静态语义检查。如果语义正确,则进行另一方面工作,即进行中间代码的翻译。这一阶段所依循的是语言的语义规则。通常使用属性文法描述语义规则。

``翻译''仅仅在这里才开始涉及到。所谓的``中间代码''是一种含义明确,便于处理的记号系统,它通常独立于具体的硬件。

\noindent\textbf{优化}

优化的的任务在于对前端产生的中间代码进行加工变换,以期在最后阶段能产生更为高效(省时间省空间)的目标代码。优化的主要方面有:公共子表达式的提取,循环优化,删除无用代码等等。有时为了便于``并行运算'',还可对代码进行并行化处理。优化所依循的原则是程序的等价变换规则。

\noindent\textbf{目标代码生成}

这一阶段的任务是:把中间代码(优化处理过后)变换成特定机器上的低级语言代码。这阶段实现了最后的翻译,它的工作有赖于硬件系统结构和机器指令含义。

目标代码的形式可以是绝对指令代码或可重定位的指令代码或汇编指令代码。如目标代码是绝对指令代码,则这种目标代码可立即执行。如目标代码是汇编指令代码,则需汇编器汇编后才能进行。

现代多数编译程序产生的是可重定位的指令代码。这种目标代码在运行前必须借助一个连接装配程序把各个目标模块(包括系统提供的库模块)连接在一起。确定程序变量(常数)在主存中的位置,装入内存中指定的起始地址,使之称为一个可运行的绝对指令代码程序。

\subsection{编译程序的结构}

\begin{figure}[H]
    \centering
    \begin{tikzpicture}[scale = 1]
        \node at (0.6,4) [text width=12pt, draw , minimum width=1.2cm , minimum height = 8cm] {表格管理};
        \node at (10.6,4) [text width=12pt, draw , minimum width=1.2cm , minimum height = 8cm] {出错处理};
    \end{tikzpicture}
    \caption{编译程序总框}
    \label{编译程序总框}
\end{figure}