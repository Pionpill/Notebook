\newpage
\section{高级语言及其语法描述}
\subsection{程序语言的定义}
\subsubsection{语法}

任何语言程序就可看成是一定字符集(字母表)上的一字符串(有限序列)。所谓的语法是指这样的一组规则,用它可以形成和产生一个合适的程序。这些规则的一部分称为词法规则,另一部分称为语法规则(产生规则)。

如:0.5 * X1 + C。可看作常数 0.5,标识符 X1,C。运算符 *,+。这些为单词符号。表达式称为语法范畴,或语法单位。

语言的单词符号是由词法规则所确定的。词法规则规定了字母表中哪样的字符串是一个单词符号。

词法规则是指单词符号的形成规则。包括各类型的常数,标识符,基本字,算符等。语法规则规定了如何从单词符号形成更大的结构(即语法单位),换言之,语法规则是语法单位的形成规则。一般程序语言的语法单位由:表达式,语句,分程序,函数等。

\subsubsection{语义}

语义问题:对于一个语言来说,不仅要给出它的词法,语法规则,而且要定义它的单词符号和语法单位的意义。

所谓一个语言的语义是指这样一组规则,使用它可以定义一个程序的意义。这些规则称为语义规则。现在还没有一种公认的形式系统,借助于它可以自动地构造出使用的编译程序,书上介绍的是基于属性文法的语法制导翻译方法。

一个程序的层次结构大体上如下:

\begin{figure}[H]
    \centering
    \begin{tikzpicture}[scale = 0.8,thick,every node/.style = {font = \small}]
        \node (程序) at (0,0) {程序};
        \node (子程序) at (0,-1.5) {子程序 \quad 或 \quad 分程序};
        \node (语句) at (0,-3) {语句};
        \node (表达式) at (0,-4.5) {表达式};
        \node (数据引用) at (-2,-6) {数据引用};
        \node (算符) at (0,-6) {算符};
        \node (函数调用) at (2,-6) {函数调用};
        \draw (程序) -- (子程序) -- (语句) -- (表达式) -- (算符);
        \draw (0,-5.2) -| (数据引用);
        \draw (0,-5.2) -| (函数调用);
    \end{tikzpicture}
    \caption{程序的层次结构}
    \label{程序的层次结构}
\end{figure}

\subsection{高级语言的一般特性}
\subsubsection{高级语言的分类}

从语言范型分类,当今的大多数程序设计语言可分为四类。
\begin{itemize}
    \item 强制式语言
    
    强制式语言也称过程式语言。其特点是命令驱动,面向语句。一个强制式语言程序由一系列的语句组成,每个语句的执行引起若干存储单元中的值得改变。
    
    \begin{C++}
        语句1;
        语句2;
        ......
        语句n;
    \end{C++}

    这类语言有:C,Pascal,FORTRAN,Ada

    \item 应用式语言
    
    应用式语言更注重程序所表示的功能,而不是一个语句接着一个语句地执行。程序的开发过程从前面已有的函数出发构造出更复杂的函数。

    \begin{C++}
        函数n(...函数2(函数1(数据))...)
    \end{C++}

    这类语言有:LISP,ML
    \item 基于规则的语言
    
    基于规则的语言程序的执行过程式:检查一定的条件,当它满足值,则执行适当的动作。也称逻辑程序设计语言。

    \begin{C++}
        条件1 -> 动作1
        条件2 -> 动作2
        ......
        条件n -> 动作n
    \end{C++}

    \item 面向对象语言
    
    面向对象语言如今已成为最流行、最重要的语言。它主要的特征是支持封装性、继承性和多态性等。把复杂的数据和用于这些数据的操作封装在一起,构成对象;对简单对象进行扩充继承简单对象的特性,从而设计出复杂的对象。通过对象的构造可以使面向对象程序获得强制式语言的有效性,通过作用于规定数据的函数的构造可以获得应用式语言的灵活性和可靠性。

    这类语言有:JAVA,C++
\end{itemize}

\subsubsection{程序结构}

一个高级语言程序通常由若干子程序段构造,许多语言还引入了类,程序包等更高级的结构。下面以说明 JAVA \footnote{FORTRAN 等语言例子见书 P16,本人没学过故不写}语言程序结构。

JAVA 是一种面向对象的高级语言,它很重要的方面是类及继承的概念,同时支持多态性和动态绑定特征。

相信各位都能了解 JAVA,这里不写了。

\subsubsection{数据类型与操作}

一个数据类型通常包括以下三种要素。
\begin{itemize}
    \item 用于区别这种类型的数据对象的属性。
    \item 这种类型的数据对象可以具有的值。
    \item 可以作用于这种类型的数据对象的操作。
\end{itemize}

\noindent\textbf{初等数据类型}

一个程序语言必须提供一定的初等类型数据成分,并定义对于这些数据成分的运算。常见的初等数据类型有:
\begin{itemize}
    \item \textbf{数值数据}:如整数,实数,复数以及这些类型的双长(或多倍长)精度数,对他们可施行算数运算(+,-,*,/等)。
    \item \textbf{逻辑数据}:多数语言有逻辑型(布尔型)数据,对它们可施行逻辑运算(and,or,not 等)。
    \item \textbf{字符数据}:字符型或字符串型数据。
    \item \textbf{指针类型}:指针的值指向另一些数据。
\end{itemize}

标识符:由字母或数字组成的以字母为开头的一个字符串。

计算机的名字仅代表一个抽象的存储单位,还必须指出它的属性。名字还包含类型和作用域等,相信读者都明白,这里不写了。

这小节内容多为数据结构的内容以及高级语言的基础知识,不写了。

\subsection{程序语言的语法描述}

在本节开始之前,首先介绍几个概念。

设 $\Sigma$ 是一个有穷字母表,它的每个元素称为一个符号。$\Sigma$ 上的一个符号串是指由 $\Sigma$ 中的符号所构成的一个有穷序列。不包含任何符号的序列被称为空字,记作 $\epsilon$。用 $\Sigma^*$ 表示 $\Sigma$ 上所有符号串的全体。

$\Sigma^*$ 的子集 U 和 V 的(连接)积定义为:
\[ UV = \{\alpha\beta | \alpha \in U \& \beta \in V\} \]
即集合 UV 中的符号串是由 U 和 V 的符号串连接而成的。V自身的 n 次(连接)积记为:
\[ V^n = \underbrace{VV...V}_{n} \]
规定 $V^0 = \{\epsilon\}$。令
\[ V^* = V^0 \cup V^1 \cup V^2 \cup V^3 \cdots \]
称 $V^{*}$ 是 V 的闭包\footnote{闭包:包含指定集合的满足在某个运算下闭合的最小集合。闭合:在一个集合上执行某种运算,得到的结果还是这个集合的元素。}。记 $V^+ = VV^*$,称 $V^+$ 是 V 的正则闭包。

闭包 $V^*$ 中的每个符号串都是由 V 中的符号串经有限次连接而成的。

\subsubsection{上下文无关法}

文法是描述语言的语法结构的形式规则(即语法规则)。所谓上下文无关法是指这样一种文法:它所定义的语法范畴(语法单位)是完全独立于这种范畴可能出现的环境的。也即遇到某个语法单位时,完全不考虑它的上下文环境,而直接进行处理(例如运算)。下文所指的文法无特别说明都是上下文无关文法。

例如我们将下面自然语言进行文法分析并绘制分析树。

\begin{center}
    He gave me a book.
\end{center}

\begin{figure}[H]
    \centering
    \begin{tikzpicture}[font = {\small}]
        \node (1-1) at (0,0) {<句子>};
        \node (2-1) at (-4,-1.5) {<主语>};
        \node (2-2) at (-1.5,-1.5) {<谓语>};
        \node (2-3) at (1,-1.5) {<间接宾语>};
        \node (2-4) at (3.5,-1.5) {<直接宾语>};
        \node (3-1) at (-4,-3) {<代词>};
        \node (3-2) at (-1.5,-3) {<动词>};
        \node (3-3) at (1,-3) {<代词>};
        \node (3-4) at (2.6,-3) {<冠词>};
        \node (3-5) at (4.4,-3) {<名词>};
        \node (4-1) at (-4,-4.5) {He};
        \node (4-2) at (-1.5,-4.5) {gave};
        \node (4-3) at (1,-4.5) {me};
        \node (4-4) at (2.6,-4.5) {a};
        \node (4-5) at (4.4,-4.5) {book};
        \foreach \x in {1,2,3,4}
            \draw (1-1) -- (2-\x) -- (3-\x) -- (4-\x);
        \draw (2-4) -- (3-5) -- (4-5);
    \end{tikzpicture}
    \caption{语法树:He gave me a book.}
    \label{语法树:He gave me a book.}
\end{figure}

一个上下文无关法包括四个组成部分:一组终结符号,一组非终结符号,一个开始符号,一组产生式。在上例中,它们分别是:

\begin{itemize}
    \item 终结符号:组成语言的基本符号 \\
    He,gave,me,a,book
    \item 非终结符号:代表语法范畴,语法概念 \\
    <句子>,<主语>,<代词> 等 
    \item 开始符号:代表所定义语言中我们最感兴趣的语法范畴 \\
    <句子>
    \item 产生式:定义语法范畴的一种书写规范\\
    <直接宾语> $\rightarrow$ <冠词> <名词>
\end{itemize}

